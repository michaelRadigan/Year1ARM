\documentclass[11pt]{article}

\usepackage{fullpage}

\begin{document}

\title{ARM Checkpoint... }
\author{Henryk Hadass \and Mickey Li \and Michael Radigan \and Oliver Wheeler}
\maketitle

\section{Group Organisation}

Our organisation was not ideal from the start. In my opinion I believe that this is because all our our group members were at different stages of learning C. We had decided to try and get the emulator finished as soon as possible. This led to a lot of work being done in the first day or two by the group member who was already familiar with C. When the rest of the group were comfortable with the language it was a real struggle to both catch up and also contribute effectively. There was a fair bit of confusion (possible exacerbated by a lack of familiarity with with git) leading ultimately to quite a bit of redundant code being written, and of course deleted. This led to much of the emulator being written by one person.

We believe that in part 2 we will be able to learn form our mistakes and manage our time more efficiently (not just because all of our group now know the language). As a result, we have all decided on the structure of the program and one member has created a skeleton files for the assembler. This will allow easier collaboration as all that needs to be done is for function bodies to be filled in. This means that we all understand how the program works, and we have a very clear structure to follow - we thus don't need to rely on one person to do all the work and experience all the difficulties that have followed.



\section{Implementation Strategies}
\subsection{Emulator}
We have based the program design has been structured on the design of a physical CPU, thus the steps involved are as so:
\begin{itemize}

\item
We have 4 main modules: \textbf{emulate.c}, \textbf{memory\_proc.c}, \textbf{cpu.c} and \textbf{instructions.c} (and their respective header files). Each one emulates an area of a physical CPU.

\item
\textbf{emulate.c} is the entry point to the application and reads in the binary file and checks for possible errors. The binary instructions are then passed on to the 'Memory' in \textbf{memory\_proc.c}

\item
\textbf{memory\_proc.c} includes all of the functions for dynamically allocating and freeing memory, little to big endian conversion, error checking and reading from memory.

\item
\textbf{cpu.c} primarily runs the fetch/execute/decode cycle with the header incorporating all the CPU and flags into a struct and the registers into an enumeration. This cycle includes the decoding which sets the relevant field in the CPU struct, and the execution which - according to the relevant opcode - returns a function pointer which executes the instruction


\end{itemize}

\subsection{Assembler}
The design of the Assembler revolves around the use of several dictionaries for lookups.

\begin{itemize}

\item 
We have 3 main modules: \textbf{assemble.c}, \textbf{encode.c} and \textbf{dictionary.c} (and their respective header files).

\item
\textbf{assemble.c} is the entry point to the program and contains the main program loop. It creates 3 lookup dictionaries - one for mapping labels to memory locations, opcodes/conditions to their binary encodings and opcodes to translation functions. 

We are using the 2-pass method to implement the assembling, so on the first pass, we read the file and note down all the label string and put them into the dictionary with their respective memory locations. On the second pass, we lookup the opcode of each line in the opcode-function lookup dictionary. This returns a function from \textbf{encode.c} which on passing the source code line to, translates it to return a binary string which is written into the output binary file.

\item
\textbf{encode.c} contains all of the functions for encoded each specific instruction into its binary representation. They all follow the same format so they can be used and stored easily in the dictionary

\item
\textbf{dictionary.c} is a polymorphic abstract data type based on a linked list. We chose a linked list as it would be more appropriate for the small data sets of this task. It is used extensively in the implementation of this assembler

\end{itemize}


\end{document}
