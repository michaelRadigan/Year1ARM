\documentclass[11pt]{article}

\usepackage{fullpage}

\begin{document}

\title{ARM Checkpoint... }
\author{Henryk Hadass , Mickey Li , Michael Radigan , Oliver Wheeler}

\maketitle

\section{Group Organisation}

\begin{enumerate}

\item
Henryk started first as he had already learnt c

\item
when others in the group got to helping a day later,they did not understand the method completely

\item 
led to a lot of confusion, and lots of redundant code being written (and deleted)

\item
in the end Henryk wrote most of the code for emulater


\item
Mickey was put in charge of writing tests for the code and tryied to introduce TDD for emulater , but ultimately failed due to time constraints. 

\item
Due to the confusion during our handling of part 1, we took a much more organised approach to the creation of the assembler. We feel that our time management and productivity will have improved as a result. 

\end{enumerate}



\section{Implementation Strategies}
\subsection{Emulator}
We have based the program design has been structured on the design of a physical CPU, thus the steps involved are as so:
\begin{enumerate}
\item
We have 4 main modules: emulate.c, memory\_ proc.c, cpu.c and instructions.c. Each one emulates an area of a physical CPU.

\item
Emulate.c is the entry point to the application and reads in the binary file and checks for possible errors. The binary instructions are then passed on to the 'Memory' in memory\_ proc.c

\item
memory\_ proc.c includes all of the functions for dynamically allocating and freeing memory, little to big endian conversion, error checking and reading from memory.

\item
cpu.c primarily runs the fetch/execute/decode cycle with the header incorporating all the CPU and flags into a struct and the registers into an enumeration. 



\end{enumerate}

\subsection{Assembler}
The design of the Assembler revolves around the use of several dictionaries for lookups.

\begin{enumerate}

\item
blah
\end{enumerate}


\end{document}
