\documentclass[11pt]{article}

\usepackage{fullpage}

\begin{document}

\title{ARM Checkpoint... }
\author{Henryk Hadass , Mickey Li , Michael Radigan , Oliver Wheeler}

\maketitle

\section{Group Organisation}

Our organisation was not ideal from the start. In my opinion I believe that this is because all our our group members were at different stages of learning C. We had decided to try and get the emulator finished as soon as possible. This led to a lot of work being done in the first day or two by the group member who was already familiar with C. When the rest of the group were comfortable with the language it was a real struggle to both catch up and also contribute effectively. There was a fair bit of confusion (possible exacerbated by a lack of familiarity with with git) leading ultimately to quite a bit of redundant code being written, and of course deleted. This led to much of the emulator being written by one person.

We believe that in part 2 we will be able to learn form our mistakes and manage our time more efficiently (not just because all of our group now know the language). 



\section{Implementation Strategies}
\subsection{Emulator}
We have based the program design has been structured on the design of a physical CPU, thus the steps involved are as so:
\begin{enumerate}
\item
We have 4 main modules: emulate.c, memory\_ proc.c, cpu.c and instructions.c. Each one emulates an area of a physical CPU.

\item
Emulate.c is the entry point to the application and reads in the binary file and checks for possible errors. The binary instructions are then passed on to the 'Memory' in memory\_ proc.c

\item
memory\_ proc.c includes all of the functions for dynamically allocating and freeing memory, little to big endian conversion, error checking and reading from memory.

\item
cpu.c primarily runs the fetch/execute/decode cycle with the header incorporating all the CPU and flags into a struct and the registers into an enumeration. 



\end{enumerate}

\subsection{Assembler}
The design of the Assembler revolves around the use of several dictionaries for lookups.

\begin{enumerate}

\item
blah
\end{enumerate}


\end{document}
